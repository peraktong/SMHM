\documentclass[ajp]{article}   	% use "amsart" instead of "article" for AMSLaTeX format
\usepackage{geometry}                		% See geometry.pdf to learn the layout mixions. There are lots.

\geometry{letterpaper}                   		% ... or a4paper or a5paper or ... 
%\geometry{landscape}                		% Activate for rotated page geometry
%\usepackage[parfill]{parskip}    		% Activate to begin paragraphs with an empty line rather than an indent
\usepackage{graphicx}				% Use pdf, png, jpg, or eps§ with pdflatex; use eps in DVI mode
								% TeX will automatically convert eps --> pdf in pdflatex		
\usepackage{amssymb}
\usepackage{amsmath, amsthm, amssymb, amsfonts}

\usepackage{dcolumn}% Align table columns on decimal point
\usepackage{bm}% bold math

\usepackage[T1]{fontenc}
\usepackage[utf8]{inputenc}
\usepackage{authblk}

\usepackage{amsmath}

\bibliographystyle{abbrv}


%SetFonts

%SetFonts


\title{Brain Storm: for $log[M_*] =12 $ Haloes}
\author[1]{Jason Cao\thanks{jc6933@nyu.edu}}
\date{}							% Activate to display a given date or no date

\begin{document}
\maketitle
\section{Reproduction}

\subsection{Reproduction of redshift-quenching model for $log[M_*] =13 $ Haloes}

\begin{center}
\includegraphics[width=170mm]{redshift_emcee_M13.png}

\centerline{Fig.1 Stellar mass vs expansion factor with redshift-quenching.}

\end{center}

\section{New Baryon conversion function}

\subsection{What we want to do}

In our former models, $f_{con}$ is just a function of $z$. What if it depends on other parameters (Like $M_h$ and $M_*$) besides $z$?

\subsection{Basic ideas}

\begin{description}
 \item[1] Ignore the quenching function for now.
 \item[2] Work on $log[M_*] =12 $ Haloes, which are still forming stars.
  \item[3] Introduce a Baryon conversion function that not only depends on z.
 \item[4] Then, find the best-fitting model under the constraint that scatter of log[M*] is smaller than 0.2 at z=0

\end{description}


\section{Example 1 -- Jeremy's $f_{con}$}

\subsection{Formula}

% log f_con = log(f0) + A1*log(M_halo[z]/Mh_threshold) + A2*log(M_star[z]/Ms_threshold) + A3*log(1+z)

\begin{equation}
log[f_{con}(z,M_h, M_*)] = log(f_0) + A_1 \cdot log[{{M_h(z)} \over{M_h^{thr}}}]+ A_2 \cdot log[{{M_*(z)} \over{M_*^{thr}}}] + A_3 \cdot log[1+z]
\end{equation}
\bigskip

The explanation of this formula is: We have an initial Baryon conversion function $f_0$. The second and the third term describe the influence of halo mass and stellar mass. I also add $M_h^{thr}$ and $M_*^{thr}$, which are the "threshold" values of halo mass and stellar mass. The reason is simple: We want to eliminate the offset term in $log(f_{con})$. If we do not eliminate this offset, it will influence the values of other terms prominently and they do have physical meanings. This won't be good. Thus, I add these terms.
But in my fitting, I fixed them at $10^{10}$ to reduce the number of variables.




\subsection{Results without the constraint that $\sigma log[M_*](z=0) <0.2$}
The best fitting model:

\begin{center}
\includegraphics[width=170mm]{brain_storm_emcee_M12.png}

\centerline{Stellar mass vs expansion factor without the constraint}

\end{center}

$f_0$ = 1.974, $A_1$ = 1.164, $A_2$ = -2.505, $A_3$ = -11.520

\bigskip


Results from MCMC chain is:
$f_0=1.528^{+0.363}_{-0.736}$, $A_1=0.111^{+0.426}_{-0.309}$, $A_2=0.744^{+0.537}_{-0.380}$, $A_3 = -2.690^{+1.522}_{1.593}$

\subsection{Results with the constraint that $\sigma log[M_*](z=0) <0.2$}
The best fitting model:

\begin{center}
\includegraphics[width=170mm]{brain_storm_emcee_M12_constraint.png}

\centerline{Stellar mass vs expansion factor with the constraint that $\sigma log[M_*](z=0) <0.2$}

\end{center}

$f_0$ = 2.008, $A_1$ = 0.610, $A_2$ = -1.881, $A_3$ = -8.213.

$\sigma log[M_*](z=0) =0.144$

\bigskip

Results from MCMC chain is:


$f_0=4.795^{+2.994}_{- 3.139}$, $A_1=0.488^{+0.295}_{-0.337}$, $A_2=0.143^{+0.144}_{-0.052}$, $A_3 = -8.689^{+1.986}_{-1.746}$


\subsection{Physics}

$f_{con}$, of course, should be related to z, $M_h$ and $M_*$. But how can we relate them. We use power law.

But the problem is our stellar mass grows too fast at the beginning.

Also, it's very tricky to decide how to add the constraint. The very first idea will be simply add it to the likelihood. If scatter >0.2, set log[likelihood] to be -infinite. But this will cause the failure of the scipy fitting.
\bigskip

Thus, I adopt the following steps.

\begin{description}
  \item[1] Fit the model without the constraint and find a set of parameters for the best-fitting model. 
  \item[2] Use these values as initial values and add the constraint. If scatter >0.2, set log[likelihood function] to be -infinite.
  \item[3] Fit the model. If scipy still can't nail it, we know the parameters around this set of parameters can't give us a scatter <0.2 result. Then, we exclude this set of parameters.
  \item[4] Repeat step 1 to 3 until we find them. (We already exclude some parameter sets.)
\end{description}
\bigskip

I think I need to add the quenching function in the next step and see what happen. Maybe the quenching function can make things better.



%%% Example 2
\section{Example 2 -- Logarithmic $f_{con}$}

\subsection{Formula}

% log f_con = log(f0) + A1*log(M_halo[z]/Mh_threshold) + A2*log(M_star[z]/Ms_threshold) + A3*log(1+z)

\begin{equation}
log[f_{con}(z,M_h, M_*)] = log(f_0) +  log[[log{{M_h(z)} \over{M_h^{thr}}}]^{A1}]+  log[[{{M_*(z)} \over{M_*^{thr}}}]^{A2}] + A3 \cdot log[1+z]
\end{equation}
\bigskip

The explanation of this formula is: We have an initial Baryon conversion function $f_0$. The second and the third term describe the influence of halo mass and stellar mass. I also add $M_h^{thr}$ and $M_*^{thr}$, which are the "threshold" values of halo mass and stellar mass. We also need to let $M_h^{thr}$ and $M_*^{thr}$ always be smaller than $M_h$ and $M_*$. Thus, we set them to be $10^{7}M_{\bigodot}$



\subsection{Physics}
The basic idea is: At big z, which means the time is small, the baryon conversion rate should almost be a constant. 

$f_con$, of course, should be related to z, $M_h$ and $M_*$. But how can we relate them. I prefer put the exponent outside the logarithm function.





Unluckily, the results seem to be bad. I'm still working on it.



reference
\bibliography{citation,pubext} 

\end{document}  